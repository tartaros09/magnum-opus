\section{Preliminaries}\label{sec:preliminaries_atomicTreeTransforms}

\begin{definition}[Ranked Alphabet]\label{def:rankedAlphabet}
    A finite set $\alphabet := \set{a_1/k_1, a_2/k_2, \dots, a_n/k_n}$ is called a \emph{finite \rab}, where $a_i$ are called the \emph{letters} and $k_i := \arity(a_i)$ are nonnegative integers called their corresponding \emph{arities}. For each nonnegative integer $k \in \N_0$, we define pairwise disjoint sets $\alphabet_k := \set{a \in \alphabet \mid \arity(a) = k} \subseteq \alphabet$. We require that $\alphabet_0 \neq \emptyset$. We also define \emph{arity} of $\alphabet$ to be $\max_i k_i$.
\end{definition}

\todoMandate{Add motivation for defining trees.}

\begin{definition}[Term, Tree]\label{def:tree}
    A \emph{term} or a \emph{tree} over a given \rab\ $\alphabet$ is a partial function $t : \N^* \to \alphabet$ with the domain $\positions(t)$ satisfying the following properties:
    \begin{itemize}
        \item $\positions(t)$ is nonempty and prefix-closed.
        \item $\forall u \in \positions(t), t(u) \in \alphabet_k, k \in \N_0 \implies \set{j \mid uj \in \positions(t)} = [k]$. \todoMandate{Rewrite with less math, more english.}
    \end{itemize}
    We say that a tree is \emph{finite} if $\positions(t)$ is finite.
\end{definition}

We shall only deal with finite trees in this dissertation, unless otherwise specified. We denote the set of all finite trees over a \rab\ $\alphabet$ by $\emph{\trees{\alphabet}}$. A \emph{tree language} over $\alphabet$ is any subset of $\trees{\alphabet}$.

\begin{definition}[Context]\label{def:context}
    A \emph{context of arity $n$} or \emph{$n$-context} over \rab\ $\alphabet$ is a tree over the \rab\ $\alphabet \uplus \set{\square_1/0, \square_2/0, \dots, \square_n/0}$ such that each letter $\square_i$ appears exactly once in it, for all $i \in [n]$. We refer to the positions occupied by letters $\square_i$ as \emph{holes} of the context for all $i \in [n]$.
\end{definition}

\todoMandate{Add motivation for regular tree languages. Reference TATA book for definition of automaton.}

\begin{definition}[Nondeterministic Top-down Tree Automaton]\label{def:ntta}
    A \emph{Nondeterministic Top-down Tree Automaton(NTTA)} over \rab\ $\alphabet$ is a tuple $\automaton := (\automatonStateSet, \alphabet, \automatonTransitionSet, \automatonInitialStateSet)$ where $\automatonStateSet$ is a finite set of \emph{states}, $\automatonInitialStateSet \subseteq \automatonStateSet$ is the set of \emph{initial states}, and $\automatonTransitionSet : \alphabet \to \relations(Q)$ is such that $\automatonTransitionSet(a) \subseteq \automatonStateSet \times \automatonStateSet^k$ is a $(k+1)$-ary relation for all $a/k \in \alphabet$. An NTTA is called \emph{Deterministic Top-down Tree Automaton(DTTA)} if for each $a/k \in \alphabet$ and $q \in \automatonStateSet$, there is a unique tuple $\vect{q} \in \automatonStateSet^k$ such that $(q, \vect{q}) \in \automatonTransitionSet(a)$.
\end{definition}

\begin{definition}[\{Run, Acceptance\} for NTTA]\label{def:runAcceptanceNtta}
    Given an NTTA $\automaton := (\automatonStateSet, \alphabet, \automatonTransitionSet, \automatonInitialStateSet)$ and a tree $t \in \trees{\alphabet}$, we define a \emph{run} of $\automaton$ on $t$ as a tree $\rho : \positions(t) \to \automatonStateSet$ which is `compatible' with $\automatonTransitionSet$, i.e., for every $u \in \positions(t)$, if $t(u) := a/k$, $\rho(u) := q$, and $\rho(ui) := q_i$, for all $i \in [k]$, then $(q, q_1, q_2, \dots, q_k) \in \automatonTransitionSet(a)$. A run $\rho$ of $\automaton$ on tree $t$ is \emph{successful} if $\rho(\e) \in \automatonInitialStateSet$. We say a tree $t$ is \emph{accepted} by NTTA $\automaton$ if there exists a successful run of $\automaton$ on $t$. The set of all trees in $\trees{\alphabet}$ accepted by $\automaton$ is called the language \emph{recognised} by $\automaton$, and denoted by $\languageOf{\automaton} \subseteq \trees{\alphabet}$. A language $L \subseteq \trees{\alphabet}$ is called \emph{regular} if there exists NTTA $\automaton$ such that $L = \languageOf{\automaton}$.
\end{definition}

\todoQuestion{Do I need to define set of all regular languages?}

\begin{definition}[Nondeterministic Bottom-up Tree \{Transition System, Automaton\}]\label{def:nbta}
    A \emph{Nondeterministic Bottom-up Tree Transition System(NBTTS)} over \rab\ $\alphabet$ is a tuple $\transitionSystem := (\automatonStateSet, \alphabet, \automatonTransitionSet)$, where $\automatonStateSet$ is a finite set of \emph{states}, and $\automatonTransitionSet : \alphabet \to \relations(\automatonStateSet)$ is such that $\automatonTransitionSet(a) \subseteq \automatonStateSet^k \times \automatonStateSet$ is a $(k+1)$-ary relation for all $a/k \in \alphabet$. A \emph{Nondeterministic Bottom-up Tree Automaton(NBTA)} is a tuple $\automaton := (\automatonStateSet, \alphabet, \automatonTransitionSet, \automatonFinalStateSet)$ such that $(\automatonStateSet, \alphabet, \automatonTransitionSet)$ is a NBTTS over $\alphabet$, and $\automatonFinalStateSet \subseteq \automatonStateSet$ is the set of \emph{final states}. An NBTA is called \emph{Deterministic Bottom-up Tree Automaton(DBTA)} if for each $a/k \in \alphabet$ and $\vect{q} \in \automatonStateSet^k$, there is a unique $q \in \automatonStateSet$ such that $(\vect{q}, q) \in \automatonTransitionSet(a)$.
\end{definition}

\begin{definition}[\{Run, Acceptance\} for NBTA]\label{def:runAcceptanceNbta}
    Given an NBTA $\automaton := (\automatonStateSet, \alphabet, \automatonTransitionSet, \automatonFinalStateSet)$ and a tree $t \in \trees{\alphabet}$, we define a \emph{run} of $\automaton$ on $t$ as a tree $\rho : \positions(t) \to \automatonStateSet$ which is `compatible' with $\automatonTransitionSet$, i.e., for every $u \in \positions(t)$, if $t(u) := a/k$, $\rho(u) := q$, and $\rho(ui) := q_i$, for all $i \in [k]$, then $(q_1, q_2, \dots, q_k, q) \in \automatonTransitionSet(a)$. A run $\rho$ of $\automaton$ on tree $t$ is \emph{successful} if $\rho(\e) \in \automatonFinalStateSet$. We say a tree $t$ is \emph{accepted} by NBTA $\automaton$ if there exists a successful run of $\automaton$ on $t$. The set of all trees in $\trees{\alphabet}$ accepted by $\automaton$ is called the language \emph{recognised} by $\automaton$, and denoted by $\languageOf{\automaton} \subseteq \trees{\alphabet}$.
\end{definition}

\todoMandate{Add result showing expressive power equivalence of NTTA, NBTA, DBTA. Also mention DTTA being weaker, with examples.}
\todoQuestion{Should there be a proof of the earlier fact?}


