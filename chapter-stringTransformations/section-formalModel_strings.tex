%%%%%%%%%%%%%%%%%%%%%%%%%%%%%%%%%%%%%%%%%%%%%%%%%%%%%%%%%%%%%%%%%%%%%%%%%%%%%%%%
%                                                                              %
%	String Transformations - Formal Model                                      %
%                                                                              %
%%%%%%%%%%%%%%%%%%%%%%%%%%%%%%%%%%%%%%%%%%%%%%%%%%%%%%%%%%%%%%%%%%%%%%%%%%%%%%%%

\section{Formal Model}\label{sec:formalModel-strings}

Let \var{} be a countable set of \emph{variables}. A \emph{pattern} is a string in $(\var \cup \Sigma)^*$. For a pattern $\VARpat$, $\var_\VARpat$ is the set of elements from $\var$ appearing in $\VARpat$, the elements of which will be called \emph{variables of \VARpat}. A \emph{valuation} is a morphism $\sigma : \var \to \Sigma^*$. By abuse of notation, we extend this to a function $\sigma : (\var \cup \Sigma)^* \to \Sigma^*$ by setting it to identity on $\Sigma$, and then naturally extending it to the asterate.  We now define \textit{actions} that transform an input text to an output text.

An \emph{action} $\alpha$ is a 4-tuple $ \alpha := (\spat, \guardL, \T, \tpat)$, where $\spat$, $\tpat$ are (source and target) patterns, $\guardL: \var_\alpha \to \reg(\Sigma)$, $\T :\var_\alpha \to \set{T : T \mbox{ is a transduction over $\alphabet$}}$ are functions. Here $\var_\alpha := \var_\spat \cup \var_\tpat$ and $\reg(\Sigma)$ is the set of regular languages over $\Sigma$.

An action $\alpha = (\spat, \guardL, \T, \tpat)$ is \emph{enabled} at a string $w$ if there exists a valuation $\ssub$ such that $\sigma(\spat) = w$ and $\sigma(x) \in \guardL(x)$ for all $x$ in  $\var_\spat$. We call $\ssub$ an \emph{enabling} valuation of $\alpha$ at $w$. We denote by $\sigma_\alpha$ the valuation defined by $\sigma_\alpha(x)=\T(x)(\sigma(x))$ for all $x$ in $\var$ (recall that $\T(x)$ is a transduction associated with $x$). The action $\alpha$ acts on $w$ using $\alpha$, resulting in the string $\sigma_\alpha(\tpat)$; we denote it by $w\cdot (\alpha, \ssub)$.

\begin{example}\label{example:gpu}
Suppose configuration information about GPUs allocated to processes are maintained in a text file. One part of  the file stores priorities of processes, using strings of the form ``process 1 : high'', ``process 2 : low'' etc. Another part of the file tracks GPUs allocated to processes, with strings of the form ``process 1 : gpu 1, gpu 2,'', ``process 2 : gpu 3,'' and so on. We describe an action that allows to move a GPU from a low priority process to a high priority one, provided there is still at least one GPU left for the low priority process. The source and target patterns are as follows:\\

\noindent
\begin{flushright}
\textbf{Source Pattern:} $x_1 \text{ process }x_2:\text{low }~x_3 
	\text{ process }x_4:\mathrm{high}~x_5~\text{ process 
}x_2:~x_6 $ ~ $ \text{ gpu } x_7 ~x_8 ~\text{ process }x_4: x_9 ~ x_{10}.$\\
%
\textbf{Target Pattern:} $x_1 \text{ process }x_2:\text{low }~x_3 
\text{ process }x_4:\mathrm{high}~x_5~\text{ process 
}x_2:~x_6 $ ~ $ x_8 ~\text{ process }x_4: x_9  \text{ gpu } x_7 ~ x_{10}$.\\
\end{flushright}

\begin{table}[t]
	\centering
\begin{tabular}{|c | l  | p{9cm}|}
	\hline
	$\var$ & $\guardL$ & Comments\\
	\hline
	\hline
	$x_1$ & $\Sigma^*$ & filler to match the prefix up to the 
	position where 	changes are to be made\\
	\hline
	$x_2$ & $[0-9]^+$ & Id of the low priority process\\
	\hline
	$x_3$ & $\Sigma^*$ & filler\\
	\hline
	$x_4$ & $[0-9]^+$ & Id of the high priority process\\
	\hline
	$x_5$ & $\Sigma^*$ & filler\\
	\hline
	$x_6$ & $(\text{gpu }[0-9]^+,)^+$ &IDs of GPUs currently allocated 
		to 	process $x_2$,  not to be transferred\\
		\hline
			$x_7$ & $[0-9]^*$ &ID of the GPU currently allocated 
		to 	process $x_2$,   to be transferred to $x_4$\\
		\hline
			$x_8$ & $\Sigma^*$ & filler\\
			\hline
				$x_9$ & $(\text{gpu }[0-9]^+,)^*$ &IDs of GPUs currently allocated 
			to 	process $x_4$\\
			\hline
			$x_{10}$ & $\Sigma^*$ & filler\\
\hline			
\end{tabular}
\caption{The variables and their guard languages from Example~\ref{example:gpu}.}\label{table:gpu}
\end{table}
The guard languages assigned and the intended purpose of the variables 
are given in Table~\ref{table:gpu}.
This action will move the gpu $x_7$ from the low priority process $x_2$ to 
the high priority process $x_4$. If changes to the configuration file 
are only allowed through this action, then GPUs cannot be moved 
from high priority to low priority processes.	
\end{example}
\smallskip

While changing strings as above, we would like to ensure that the 
syntactical structure of the strings is not broken. Suppose the 
contents of the file in the above example belong to the language 
$(\mathrm{process [0-9]^+}:(\mathrm{high}+\mathrm{low}), )^* 
(\mathrm{process}[0-9]^+:(\mathrm{gpu}[0-9]^+,)^*)^*$. We want to verify that after applying the actions, the resulting string is still in the 
language. We formalise this next.

Let $\Sigma$ be a finite alphabet, $L \subseteq \Sigma^*$ be a language and $\alpha$ be an action over $\Sigma$. We denote by
$\post(L, \alpha)$ the set $\{w \cdot (\alpha, \ssub) \mid  w \in L, \ssub \text{ enables } \alpha \text{ at } w\}$ of results of $\alpha$ acting on strings in $L$. We study the following problem:
\medskip

\uline{\emph{Invariance-checking Problem}}

\textbf{Input:} Action $\action$, regular language $L$.

\textbf{Question:} $\post(L, \action) \subseteq L$?

\begin{example}
	Consider an action $\alpha$  with $\spat := xy, \tpat := yx$, $\guardL$ assigns $\Sigma^*$  and $\T$ assigns the identity transduction
	to all variables. Note that $\post(L,\alpha)=\set{uv \mid vu \in L, u, v \in \Sigma^*}$ is the \emph{rotational closure} of $L$. A
	given language $L$ is invariant under $\alpha$ only if $L$ is closed under rotations.
\end{example}

We study the complexity of the invariance-checking problem in the rest of the paper.
