\section{Abstract}\label{sec:abstract-trees}

We introduce and study pattern-based tree transformations. As an illustrating example, consider a source pattern $(x \cdot y) + (x \cdot z)$ and a target pattern $x \cdot (y + z)$ as a pair. This source pattern matches any expression $e$ of the form $(e_1 \cdot  e_2) + (e_1 \cdot  e_3)$ (by substituting $x$ with $e_1$, $y$ with $e_2$ and $z$ with $e_3$) and the pair transforms it into the expression $e_1 \cdot (e_2 + e_3)$ as dictated by the target pattern. Note that in this example, the set of expressions that matches the source pattern is not a regular tree language, due to the repetition of the exactly identical expression $e_1$ in both the parentheses.

We propose a model of tree transformations given by a finite representation of a (possibly infinite) set of such (source pattern, target pattern) pairs. The expressive power of this model comes at the cost of undecidability of checking equivalence. Nevertheless, we show that the type-checking problem is decidable for our model of pattern-based tree transformations. The type-checking problem asks whether applying a given transformation to trees having a given regular property (type) preserves the property. Our decision procedure is by a reduction to the emptiness problem of alternating tree automata.   
