%%%%%%%%%%%%%%%%%%%%%%%%%%%%%%%%%%%%%%%%%%%%%%%%%%%%%%%%%%%%%%%%%%%%%%%%%%%%%%%%
%                                                                              %
%	Chapter 03, Section 07 - Conclusions                                       %
%                                                                              %
%%%%%%%%%%%%%%%%%%%%%%%%%%%%%%%%%%%%%%%%%%%%%%%%%%%%%%%%%%%%%%%%%%%%%%%%%%%%%%%%

\section{Conclusions}\label{sec:c03-conclusions}

We have introduced a formal model for fine-controlled text modifications. We have shown that the regular invariance-checking problem is \pspc. For a restriction, the problem is shown to be \conpc.

It is interesting to see whether we can lift our results to other transducer models inside an action, instead of Mealy machines. An important next step is to see whether we can have actions that act on trees or structured text instead of simple texts. Trees are the main structures used to store configuration data by Apache ZooKeeper, one of our motivations for the work. Further it could also serve as a syntax checker for code translators etc., if we have structured text (source codes, XML). It is worth investigating whether we can lift the actions to form a sort of  visibly pushdown transducer \cite{FRRST2018}, and whether visibly pushdown languages \cite{AM2004} can be checked for invariance.
