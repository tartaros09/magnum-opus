%%%%%%%%%%%%%%%%%%%%%%%%%%%%%%%%%%%%%%%%%%%%%%%%%%%%%%%%%%%%%%%%%%%%%%%%%%%%%%%%
%                                                                              %
%   Tree Transformations - Problems and Results                                %
%                                                                              %
%%%%%%%%%%%%%%%%%%%%%%%%%%%%%%%%%%%%%%%%%%%%%%%%%%%%%%%%%%%%%%%%%%%%%%%%%%%%%%%%

\section{Problems and Results}\label{sec:problemsAndResults}

A simple question is to check whether there is a matching.  This problem is already NP complete. \\
\noindent
\probdef{\matchingProblemFull\ }
{Pattern $\pattern$, tree $t$, guard $\guardFunction$. {\color{gray} \slash\slash The guard languages are represented by NBTA }}
{Does $t$ match $\pattern$ under $\guardFunction$?}

\begin{theorem}[\cite{mtchLBound}]\label{thm:matching}
    \matchingProblemFull\ is \nptc. %, provided the guard languages are
    specified as NBTA.
\end{theorem}
\begin{proof}
        \textbf{(Upper Bound)} Let $(\pattern, t)$ be an instance of
        \matchingProblemFull, where $\pattern$ is a pattern over
        $\Alphabet$ and
        $\Variables$, $t \in
        \Trees{\Alphabet}$ and $\guardFunction : \Variables \to 2^{\Trees{\Alphabet}}$ where each value of the function is given by an NBTA. Let the size of the input be $N :=
        |\pattern| + |t| + |\guardFunction|$. Given a certificate, which is a particular
        substitution whose size is bounded by $|t|$, we can verify whether $\substitution(\pattern) =
        t$ in $\bigo(|t|)$ time, which is linear in the size of the
        input $N$. We can also verify that for every $x \in
        \Variables(\pattern)$, $\substitution(x) \in
        \guard{x}$. Since there exists a polynomial-time verifier
        algorithm for the problem, we see that \matchingProblemFull\
        is in \np.
	
	\textbf{(Lower Bound)} For this proof, we refer to~\cite{mtchLBound},
        which gives the proof in the case of words. The proof proceeds
        by obtaining a reduction from \oneInThreeSatFull. For the sake
        of completeness we include a version of the proof in the Appendix~\ref{sec:mtchLB}.
\end{proof}

A problem which we consider for atomic transformations is the following:

\probdef{\typecheckingProblemFull}
{Atomic transformation $\atomicTransform := (\sourcePattern, \targetPattern, \guardFunction)$, source language $\sourceLang$,  target language $\targetLang$. {\color{gray} \slash\slash The  languages are represented by NBTA }}
{Do we have that $\atomicTransform[\sourceLang] \cap \targetLang \neq \emptyset$?}

For this, we have the following results:
\begin{theorem}\label{thm:typecheckingProblem}
    \typecheckingProblemFull\ is \exptc.
\end{theorem}

A detailed proof can be found in Appendix~\ref{sec:atomicTypeCheckingProof}. 

We next turn to the main focus of study, the type-checking problem:\\
\probdef{\metaTypecheckingProblemFull\ }{ Transformation  $\fullTransform := (\seedLang,
	\sourceMap, \targetMap, \guardFunction)$, source language $\sourceLang$,  target language $\targetLang$. {\color{gray} \slash\slash The  languages are represented by NBTA }}
{Do we have that $\fullTransform[\sourceLang] \cap \targetLang \neq \emptyset$?}
%\ais{Ideally input should be transformation, L1, L2}
%
%%\subsection{Results}\label{subsec:results}
%\aisinline{say that why our example of type-checking from before fits into this definition.}

Following is the main result of this paper.
\begin{theorem}\label{thm:metaTypecheckingUpperBound}
  \metaTypecheckingProblemFull\ is in 2-EXPTime. In particular,  it can be solved in time
  $2^{\mathsf{polynomial}(\mathsf{input})}$, where the degree of the
  polynomial is linear in the size of the input. 
\end{theorem}
%\ais{poly instead of polynomial ?}
%\ais{this sounds dishonest. why do not we say double exponential?}

Our decision procedure for the type-checking problem is automata theoretic. In fact, we reduce the type-checking problem to the non-emptiness-checking of an alternating tree-automata that we construct