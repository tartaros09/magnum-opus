%%%%%%%%%%%%%%%%%%%%%%%%%%%%%%%%%%%%%%%%%%%%%%%%%%%%%%%%%%%%%%%%%%%%%%%%%%%%%%%%
%                                                                              %
%   Tree Transformation - Conclusion                                           %
%                                                                              %
%%%%%%%%%%%%%%%%%%%%%%%%%%%%%%%%%%%%%%%%%%%%%%%%%%%%%%%%%%%%%%%%%%%%%%%%%%%%%%%%

\section{Conclusion}\label{sec:conclusion-trees}

We have introduced a formal model of tree transformations which is highly expressive. We study the type-checking problem and give an automata-theoretic algorithm.
It is interesting to look at other variants where the (source pattern, target pattern) pair is generated by more generic versions of tree transducers. 
