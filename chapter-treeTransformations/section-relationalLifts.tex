%%%%%%%%%%%%%%%%%%%%%%%%%%%%%%%%%%%%%%%%%%%%%%%%%%%%%%%%%%%%%%%%%%%%%%%%%%%%%%%%
%                                                                              %
%   Tree Transformation - Proof Outline                                        %
%                                                                              %
%%%%%%%%%%%%%%%%%%%%%%%%%%%%%%%%%%%%%%%%%%%%%%%%%%%%%%%%%%%%%%%%%%%%%%%%%%%%%%%%

\section{Relational Lifts}\label{sec:relationalLifts}

Suppose $\automaton$ is a word automaton with set of states $Q$ and transition relation given by $\Transitions: \Alphabet \to \Relations(Q,2)$, we can extend it to $\widetilde{\Transitions}: \Alphabet^{*} \to \Relations(Q,2)$ such that $(q_1,q_{2}) \in \widetilde{\Transitions}(w)$ iff $\automaton$ has a run on the word $w$ starting from $q_{1}$ and ending at $q_{2}$. The following definition describes a similar extension for NBTTS.

\begin{definition}[Extended Transition Relation(\extendedTransitionRelation)]\label{def:extendedTransitionRelation}
  Let $\T := (Q, \Alphabet, \Transitions)$ be a NBTTS. We extend the function $\Transitions$ to a larger function $\widetilde{\Transitions} : \Contexts{\Alphabet}{} \to \Relations(Q)$ called the \emph{Extended Transition Relation} or \emph{\extendedTransitionRelation} as follows. Let $c \in \Contexts{\Alphabet}{k}$. For a tuple $\q := (q_1, q_2, \dots, q_k)$, define the NBTTS $\T_{\q} := (Q, \Alphabet\uplus\{\square_1/0, \square_2/0, \dots, \square_k/0\}, \Transitions\uplus\{(\square_i, \{q_i\}) \mid i \in [k] \})$. Let $\Results(\q) := \{(\q, q) \in Q^{k+1} \mid q \mbox{ is the result of a run of $\T_{\q}$ on $c$.}\}$. Then we define $\widetilde{\Transitions}(c) := \bigcup_{\q \in Q^k} \Results(\q)$. We also refer to the relation $\widetilde{\Transitions}(c)$ as the \emph{relation induced} by the context $c$ on the states of the NBTTS $\T$. Note that for $a \in \Alphabet$, we have that $\widetilde{\Transitions}(a) = \Transitions(a)$.
\end{definition}

Suppose a word $w_{1}$ is extended by appending another word $w_{2}$ to its end. Now, $\widetilde{\Transitions}(w_{1} \cdot w_{2})$ is obtained by simply composing the relations $\widetilde{\Transitions}(w_{1})$ and $\widetilde{\Transitions}(w_{2})$. For trees, it is more complicated: a tree $t_{1}$ can have multiple ``end points'' where it can be extended by appending other trees. We will need a more complicated way of composing relations.

\begin{definition}[Relational Lifts]\label{def:relationalLift}
  Let $X$ be a set and $n \in \N\setminus\{0\}$ be a nonnegative integer. Let $\operator \subseteq X^{n+1}$ be a relation. We define the \emph{relational lift} of $\operator$ as a new relation $\widetilde{\operator} \subseteq \Relations(X)^{n+1}$ given as follows: For $R_1, \dots, R_n, R \in \Relations(X)$, we say that $(R_1, R_2, \dots, R_n, R) \in \widetilde{\operator}$ if and only if
  \begin{itemize}
    \item $R \in \Relations\left(X, 1+\sum_{j=1}^n i_j\right)$, where $i_j \in \N\setminus\{0\}$ for all $j \in [n]$ are such that $R_j \in \Relations(X, i_j+1)$.
    \item For all $j \in [n]$, for all $\x_j \in X^{i_j}$, we have $(\x_1, \x_2, \dots, \x_n, y) \in R$ if and only if there exist $y_1, y_2, \dots, y_n \in X$ such that $(\x_j, y_j) \in R_j$ and $(y_1, y_2, \dots, y_n, y) \in \operator$.
  \end{itemize}    
\end{definition}

We can think about relational lift as a procedure to compose arbitrarily many relations. Another, more abstract way to think about it is as a way of ``lifting'' an operator over a set to an operator over relations over that same set. In case of automata running on words, for any $a \in \Alphabet$ and any $w_{1} \in \Alphabet^{*}$, if we set $\operator := \Transitions(a)$, then $(\widetilde{\Transitions}(w_{1}), \widetilde{\Transitions}(w_{1}\cdot a)) \in \widetilde{\operator}$. In case of NBTTS running on contexts, suppose $c_{1}, \ldots, c_{k}$ are the children of a node labeled by the letter $a/k$ and we set $\operator := \Transitions(a) \subseteq Q^{k+1}$, then $(\widetilde{\Transitions}(c_{1}), \ldots, \widetilde{\Transitions}(c_{k}), R) \in \widetilde{\operator}$ implies that $R = \widetilde{\Transitions}(a(c_{1}, \ldots, c_{k}))$.

%Note that $\widetilde{\Transitions}|_{\Alphabet} = \Transitions$. Intuitively, $\widetilde{\Transitions}$ extends $\Transitions$ such that the following diagram commutes.
%    \[\begin{CD}
%        \Alphabet @>>> \Trees{\Alphabet}\\
%        @VV\Transitions V @VV\widetilde{\Transitions}V\\
%        \Relations(Q) @>\sim>> \Relations(\Relations(Q))
%    \end{CD}\]

The following problem and its complexity is an important intermediate technical lemma used later.

\noindent
\probdef{\relationRealisabilityFull}
{NBTTS $\T := (Q, \Alphabet, \Transitions)$, relation $R \in \Relations(Q)$, NBTA $\D$.}
{Does there exist $c \in \Contexts{\Alphabet}{} \cap L(\D)$ such that $\widetilde{\Transitions}(c) = R$?}

As a matter of nomenclature, if the above condition holds for a given $\T, R, \D$, then we say that the relation $R$ is \emph{realisable} by $\T$ under $\D$.

\begin{theorem}\label{thm:relationRealisabilityExptc}
  \textsc{\relationRealisabilityFull}\ is \exptc.
\end{theorem}

The result follows from a reduction from the Intersection Nonemptiness problem for DBTA. A detailed proof of the same can be found in Appendix~\ref{sec:relationRealisabilityProof}

In the proof for the above result, we also get that for each relation which we are checking for realisability, we can also get, in \expt, a context over the alphabet which realises it(if it is realisable). In such a case, we will say that the relation $R$ is \emph{realised} by the context $c$ on the NBTTS $\T$, under $\D$. We also say that $c$ \emph{induces} $R$ on $\T$ under $\D$.  For a given pair $(\T, \D)$ with NBTTS $\T$ and NBTA $\D$, we define the set
\[ \RealisableRelations{\T}{\D} := \{(R, c) \mid \mbox{ $R$ is realised by $c$ on $\T$ under $\D$}\} \]