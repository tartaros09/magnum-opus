\section{Formal Model}\label{sec:formalModel-trees}

\subsection{Preliminaries}\label{subsec:preliminaries}

Let $\N := \set{1, 2, 3, \dots}$ be the set of strictly positive integers and let $\N_0 := \N \cup \{0\}$. For $n \in \N$, we denote the set $\{1, \ldots, n\}$ by $\counting{n}$. As a convention, we let $\counting{0} := \emptyset$. For a set $X$ and $n \in \N$, $X^n$ denotes the usual Cartesian power of $X$, consisting of all $n$-tuples of elements in $X$. As a convention, we let $X^0 := \{\natural\}$, a special singleton set.

%\begin{definition}[Cartesian Power of a Set]\label{def:setPower}
%    Let $X$ be a set. For a positive integer $n \in \N$, we define the %\emph{$n$-ary Cartesian power}, or just the \emph{$n$-ary power %of} $X$ to be the set $X^n := \{(x_1, x_2, \dots, x_n) \mid x_i \in X, %\forall i \in [n]\}$. As a part of convention, we also define $X^0 := %\{\natural\}$, a special singleton set, for all sets $X$. We also define %$X^* := \bigcup_{k=0}^\infty X^k$.
%\end{definition}

For $n \in \N_0$, an $n$-ary relation over $X$ is a subset $R \subseteq X^n$. We let $\relations{X}{n}$ denote the set of all $n$-ary relations on $X$. Finally, we denote $\bigcup_{k=0}^\infty \relations{X}{k}$ by $\relations{X}{}$.

%\begin{definition}[Relations on a Set]\label{def:relation}
%    Let $X$ be a set. For a nonnegative integer $n \in \N\setminus\{0\}$, a \emph{relation} of arity $n$, or an \emph{$n$-ary relation} over $X$ is defined as subset $R \subseteq X^n$. We also define $\Relations(X, n)$ to be the set of all $n$-ary relations on $X$. Finally, we define $\Relations(X) := \bigcup_{k=0}^\infty \Relations(X, k)$.
%\end{definition}

We consider automata over finite ranked trees, broadly following definitions from \cite{tata}.

\begin{definition}[Alphabet]\label{def:alphabet}
    A finite set $\alphabet := \{a_1/k_1, \dots, a_n/k_n\}$ is called a \emph{finite ranked alphabet} or \emph{alphabet}, where $a_1, \dots, a_n$ are called the \emph{letters}, and $k_1, \dots, k_n \in \N_0$ are nonnegative integers called their corresponding \emph{arities}. We define pairwise disjoint sets $\alphabet_k := \{a \in \alphabet \mid \mbox{arity of $a$ is $k$}\} \subseteq \alphabet$ for all $k \in \N_0$. We require that $\alphabet_0 \neq \emptyset$. We also define the arity of $\alphabet := \max_{i} k_i$.
\end{definition}

\begin{definition}[Term, Tree]\label{def:tree}
    A \emph{term} or a \emph{tree} over a ranked alphabet $\alphabet$ is a partial function $t : \N^* \to \alphabet$ with the domain $\positions(t)$ satisfying the following properties:
    \begin{itemize}
        \item $\positions(t)$ is nonempty and prefix-closed.
        \item $\forall u \in \positions(t), t(u) \in \alphabet_k, k \in \N_0 \implies \{j \mid uj \in \positions(t)\} = \counting{k}$.
    \end{itemize}
    We say that a tree $t$ is \emph{finite} if $\positions(t)$ is finite.
\end{definition}
In this paper, we will only deal with finite trees. We denote the set of all finite trees over $\alphabet$ by $\trees{\alphabet}$. A \emph{tree language} is any subset of $\trees{\alphabet}$. Regular tree languages are those recognised by tree automata as defined in the following.

\begin{definition}[Alternating Top-down Tree Automata]\label{def:alternatingTreeAutomaton}
    A \emph{alternating top-down tree automaton} or \emph{ATTA} over alphabet $\alphabet$ is a tuple $\automaton := (\automatonStateSet, \alphabet, \automatonInitialStateSet, \automatonTransitionSet)$, where $Q$ is a finite set of \emph{states}, $\automatonInitialStateSet \subseteq Q$ is a set of initial states, and $\automatonTransitionSet : \automatonStateSet \times \alphabet \to \B^+(\automatonStateSet \times \N)$ such that $\automatonTransitionSet(q, a/k) \in \B^+(\automatonStateSet \times \counting{k})$, for all $q \in \automatonStateSet, a/k \in \alphabet$. Here, for any set $X$, $\B^+(X)$ denotes the set of all positive boolean combinations of elements in $X$. An ATTA over $\alphabet$ is called a \emph{nondeterministic (resp.~deterministic) top-down tree automaton} or \emph{NTTA} (resp.~\emph{DTTA}) if for all $q \in \automatonStateSet, a/k \in \alphabet$, $\automatonTransitionSet(q, a/k) = \bigvee_{(q_1, \dots, q_k) \in S} \bigwedge_{i=1}^{k} (q_i, i)$, for some $S \subseteq \automatonStateSet^k$ (resp.~$S \subseteq \automatonStateSet^k, \abs{S} = 1$).
\end{definition}

The semantics of ATTA are defined using runs, which are themselves trees satisfying some conditions.
\begin{definition}[Run, Acceptance for ATTA]\label{def:acceptanceByAtta}
    Given a tree $t \in \trees{\alphabet}$ and an ATTA $\automaton$ over $\alphabet$, we define a \emph{run} of $\automaton$ on $t$ to be a tree $\rho$ over $\automatonStateSet \times \N^*$ such that $\rho(\epsilon) = (q, \epsilon)$ for some state $q$ and every position $u \in \positions(\rho)$ satisfies the following condition: if $\rho(u) = (q, x), t(x) = a/k, \automatonTransitionSet(q ,a) = \phi$, then there is a subset $S := \{(q_1, i_1), \dotsm (q_n, i_n)\} \subseteq Q \times \counting{k}$ such that $S \models \phi$, the successor positions of $u$ in $\rho$ are $\{u1, \dots, un\}$, and $\rho(uj) = (q_j, xi_j)$ for all $j \in \counting{n}$.
    
    A run is \emph{successful} if $\rho(\epsilon) = (q, \epsilon)$ for some initial state $q \in I$. A tree $t$ is \emph{accepted} by ATTA $\automaton$ if there exists at least one successful run of $\automaton$ on $t$. The set of all trees accepted by ATTA $\automaton$ is called the tree language \emph{recognised} by $\automaton$, and denoted by $\languageOf{\automaton} \subseteq \trees{\alphabet}$. A tree language $L \subseteq \trees{\alphabet}$ is called \emph{regular} if there exists an ATTA $\automaton$ such that $L = \languageOf{\automaton}$. The set of all regular tree languages over $\alphabet$ is denoted by $\regular{\alphabet} \subseteq 2^{\trees{\alphabet}}$.
\end{definition}

Specifying guard languages using bottom-up tree automata turns out to be technically more convenient.
\begin{definition}[Nondeterministic Bottom-up Tree \{Transition System, Automaton\}]\label{def:nbtts}
    Let $\alphabet$ be a finite alphabet. A \emph{Nondeterministic Bottom-up Tree Transition System} or \emph{NBTTS} over $\alphabet$ is defined as $\T := (Q, \Alphabet, \Transitions)$, where $Q$ is a finite set of \emph{states}, and $\Transitions : \Alphabet \to \relations{Q}{}$ is a function such that $\Transitions(a/k) \in \relations{Q}{k+1}$, for all $a/k \in \Alphabet$. A \emph{Nondeterministic Bottom-up Tree Automaton} or a \emph{NBTA} is a tuple $\automaton := (Q, \Alphabet, \Transitions, \automatonFinalStateSet)$, where $(Q, \Alphabet, \Transitions)$ is a NBTTS over $\Alphabet$ and $\automatonFinalStateSet \subseteq Q$ is a set of \emph{final} states. If for each $a/k \in \Alphabet$ we have that $\Transitions(a/k)$ is such that for all $q_1, \dots, q_k \in Q$, there exists a unique $q \in Q$ such that $(q_1, \dots, q_k, q) \in \Transitions(a)$, then the corresponding NBTTS is called a \emph{Deterministic Bottom-up Tree Transition System} or \emph{DBTTS}, and the corresponding NBTA is called a \emph{Deterministic Bottom-up Tree Automaton} or \emph{DBTA}. For a given alphabet $\Alphabet$, let the set of all $NBTA$ over $\Alphabet$ be denoted by  $\Automata{\Alphabet}$.
\end{definition}

\begin{definition}[\{Run, Result, Acceptance\} for NBTA]\label{def:acceptanceByNbta}
    Given a NBTTS $\T := (Q, \Alphabet, \Transitions)$ and a tree $t \in \Trees{\Alphabet}$, a \emph{run} of $\T$ on $t$ is defined as a tree $\rho : \Positions(t) \to Q$ which is `compatible' with $\Transitions$, i.e., for every position $u \in \Positions(t)$, if $t(u) = a/k, \rho(u) = q$ and $\rho(ui) := q_i$ for all $i \in \counting{k}$, then $(q_1, \dots, q_k, q) \in \Transitions(a)$. The \emph{result} of a run $\rho$ is $\result(\rho) := \rho(\epsilon)$. Given a NBTA $\automaton := (Q, \Alphabet, \Transitions, \automatonFinalStateSet)$, and a tree $t \in \Trees{\Alphabet}$, we say that $t$ is \emph{accepted} by $\automaton$ if and only if there exists a run $\rho$ of $(Q, \Alphabet, \Transitions)$ on $t$ such that $\result(\rho) \in \automatonFinalStateSet$. The set of all trees $t$ accepted by a NBTA $\automaton$ is called the language \emph{recognised} by $\automaton$, and is denoted by $\languageOf{\automaton} \subseteq \Trees{\Alphabet}$.
\end{definition}

Even though we have introduced a few different automata, that is merely for ease of use. From \cite{tata}, we have that in terms of expressive power, all the automata in the set $\set{ATTA, NTTA, NBTA, DBTA}$ are identical, i.e., they all have the exact same expressive power. On the contrary, a $DTTA$ is strictly weaker than any automata mentioned earlier in this paragraph.

Just like a word $w$ can be extended by appending another word to its end, a tree can be extended by appending other trees to designated leaves. This is formalised below, where $ \uplus$ denotes disjoint union.

\begin{definition}[Context]\label{def:context}
    A \emph{context with arity $n$} or an \emph{$n$-context} over $\Alphabet$ is a tree over the alphabet $\Alphabet \uplus \{\square_1/0, \dots, \square_n/0\}$ such that each letter $\square_i$ appears exactly once in the tree, for all $i \in \counting{n}$. We refer to the positions occupied by $\square_i$ as the \emph{holes} of the context for all $i \in \counting{n}$.
\end{definition}

In this paper, we shall only deal with finite contexts. Intuitively, the leaves labeled with $\square_1, \dots, \square_n$, or the holes of the context are the designated positions where other trees can be appended. We denote the set of all finite $n$-contexts over $\Alphabet$ by $\Contexts{\Alphabet}{n}$. We denote $\bigcup_{n=0}^\infty \Contexts{\Alphabet}{n}$ by $\Contexts{\Alphabet}{}$, where we let $\Contexts{\Alphabet}{0} := \Trees{\Alphabet}$.

\subsection{Tree Transformations}\label{subsec:model}

In this section, we define our tree transformation model and describe some associated problems that we wish to study. 

We let $\Alphabet$ be a finite ranked alphabet. We will need trees, some parts of which are not completely specified and we use variables to denote those parts. Let $\Variables$ be a countable set of \emph{ranked variables} or \emph{variables}. For a variable $x \in \Variables$, $k_x \in \N_0$ denotes its arity. We define pairwise disjoint sets $\Variables_k := \{x \in \Variables \mid \mbox{arity of $x$ is $k$}\} \subseteq \Variables$ for all $k \in \N_0$.

\begin{definition}[Pattern]\label{def:pattern}
    A \emph{tree pattern} or simply a \emph{pattern} over $\Alphabet$ and $\Variables$ is defined to be a tree $\pattern \in \Trees{\Alphabet \uplus \Variables}$. We also let \emph{$\variablesOf{\pattern}$} to be the set of all variables appearing in $\pattern$. Note that $\variablesOf{\pattern} \subseteq \Variables$ is necessarily finite.
\end{definition}

A pattern can represent multiple trees by substituting variables  in the pattern by contexts of the same arity.
\begin{definition}[Substitution]\label{def:substitution}
    A \emph{substitution} over $\Alphabet$ is a partial function $\substitution : \Variables \to \Contexts{\Alphabet}{}$ such that for all $x \in \dom(\substitution)$, we have $\substitution(x) \in \Contexts{\Alphabet}{k}$, where $k$ is the arity of $x$.
\end{definition}

Given a substitution $\substitution$, we can naturally extend it to the set of patterns over $\Alphabet$ as follows. Define a tree homomorphism $\widetilde{\substitution} : \Trees{\Alphabet \uplus \dom(\substitution)} \to \Trees{\Alphabet}$ by setting
\[\twopartfunc{\widetilde{\substitution}(x)}{x}{x \in \Alphabet}{\substitution(x)}{x \in \dom(\substitution)}\]
and extending it homomorphically. Since $\widetilde{\substitution}|_{\dom(\substitution)} = \substitution$, by abuse of notation we shall write $\widetilde{\substitution}$ as $\substitution$ in places where the distinction is irrelevant.

We would like to control the set of contexts that can be substituted for variables in patterns, for which we use guards. A \emph{guard} over a finite alphabet $\Alphabet$ is a partial function $\guardFunction : \Variables \to 2^{\Trees{\Alphabet}}$, which assigns to each variable in its domain, a tree language over $\Alphabet$. The following formally defines when a pattern matches a tree.
\begin{definition}[Match]\label{def:match}
    Let $\pattern$ be a pattern over $\Alphabet$ and $\Variables$ and let $t \in \Trees{\Alphabet}$ be a tree. Let $\guardFunction$ be a guard such that $\variablesOf{\pattern} \subseteq \dom(\guardFunction)$. We say that $t$ \emph{matches} pattern $\pattern$ under $\guardFunction$ if and only if there exists a substitution $\substitution$ such that
    \begin{itemize}
        \item $\variablesOf{\pattern} \subseteq \dom(\substitution) \subseteq \dom(\guardFunction)$.
        \item $\substitution(x) \in \guard{x}$ for each $x \in \variablesOf{\pattern}$.
        \item $\substitution(\pattern) = t$.
    \end{itemize}
    In such a case, $\substitution$ is called a \emph{matching substitution}, and we say the substitution $\substitution$ \emph{matches} the tree $t$ to the pattern $\pattern$ under the guard $\guardFunction$.
\end{definition}

There may exist more than one matching substitution for a given match. Eg., if $a/1, b/1, c/0$ are letters and $x/1, y/0$ are variables, the tree $a-b-b-c$ matches the pattern $x-b-y$ with 2 distinct substitutions $\substitution_1, \substitution_2$ given by $\substitution_1(x) := a, \substitution_1(y) := b-c$ whereas $\substitution_2(x) := a-b, \substitution_2(y) := c$.

The following formalises the tree transformations defined by our model.
\begin{definition}[Atomic Transformation]\label{def:atomicAction}
    Let $\Alphabet$ be a finite alphabet and $\Variables$ be a countable set of variables. An \emph{atomic tree transformation} or simply, an \emph{atomic transformation} is an ordered triple $\atomicTransform := (\sourcePattern, \targetPattern, \guardFunction)$, where $\sourcePattern, \targetPattern$ are patterns over $\Alphabet$ and $\Variables$ called the \emph{source pattern} and \emph{target pattern} respectively, and $\guardFunction$ is a guard over $\Alphabet$ such that $\variablesOf{\sourcePattern} \cup \variablesOf{\targetPattern} \subseteq \dom(\guardFunction)$. The \emph{relation over trees induced by an atomic transformation} is
    \begin{eqnarray*}
        \relT(\atomicTransform)
        & := & \{(t,t') \mid \exists \substitution \mbox{ s. t. } \substitution \mbox{ matches } t  \mbox{ to } \sourcePattern \mbox{ under } \guardFunction\\
        &   & \mbox{ and } \substitution \mbox{ matches } t'  \mbox{ to } \targetPattern \mbox{ under } \guardFunction \}
    \end{eqnarray*}
\end{definition}
We denote by $\atomicTransform(t) = \{t' \mid (t,t') \in \relT(\atomicTransform)\}\subseteq \Trees{\Alphabet}$ the \emph{effect of $\atomicTransform$} on the tree $t$. By abuse of notation, we naturally extend the effect of $\atomicTransform$ to tree languages $L$ by setting $\atomicTransform(L) := \cup_{t \in L}\atomicTransform(t)$. Our model consists of a possibly infinite set of atomic transformations, represented by a bimorphism.

\begin{definition}[Transformation]\label{def:action}
    Let $\Alphabet, \seedAlphabet$ be two finite alphabets. Let $\Variables$ be a countably infinite set of variables. An \emph{transformation} $\fullTransform$ is a tuple $\fullTransform := (\seedLang, \sourceMap, \targetMap, \guardFunction)$ where
    \begin{itemize}
        \item $\seedLang \subseteq \Trees{\seedAlphabet}$ is called the \emph{seed language}.
        \item $\sourceMap, \targetMap : \Trees{\Gamma} \to \Trees{\Alphabet \uplus \Variables}$ are tree homomorphisms.
        \item $\guardFunction$ is a guard over $\Alphabet$.
    \end{itemize}
    We define the relation induced by a transformation $\fullTransform := (\seedLang, \sourceMap, \targetMap, \guardFunction)$ as follows: for each seed tree $s \in \seedLang$, we define the atomic transformation $\atomicTransform_s:= (\sourceMap(s), \targetMap(s), \guardFunction)$. Then
    \[ \relT(\fullTransform) := \bigcup_{s \in \seedLang}\relT(\atomicTransform_s) \]
\end{definition}
For a tree $t$ and a tree language $L$, $\fullTransform(t) = \{t' \mid (t,t') \in \relT(\fullTransform)\}\subseteq \Trees{\Alphabet}$ and $\fullTransform(L) := \cup_{t \in L}\fullTransform(t)$ as before.
